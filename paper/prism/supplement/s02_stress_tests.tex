\section{Stress Tests}
\label{supp:stress-tests}

\sectionpurpose{This section records the stress mechanisms used to challenge calibration transport, correction stability, and posterior robustness under adversarial or low-support regimes.}

\sectionlearn{
\item How adversarial timing and liquidity stresses are parameterized.
\item Why misspecification grids are needed for fusion robustness checks.
\item Which stresses are most likely to trigger diagnostic failures.
}

Stress mechanisms include:
\begin{itemize}[leftmargin=*]
\item timing attacks that concentrate directional flow in short windows,
\item low-liquidity settings where a small number of messages dominate weighted evidence,
\item mismatch injections between calibration assumptions and generated outcomes,
\item regime-switch windows where correction parameters drift relative to training periods.
\end{itemize}

For each mechanism, PRISM logs whether correction maps amplify or attenuate instability. The intended interpretation is diagnostic: stress failures identify boundary regions where assumptions do not hold strongly enough for aggressive claims.

Current scenario-level audit artifacts indicate substantial stress sensitivity in several families. In particular, no-regret and concentration-oriented claims are not uniformly supported under low-support and coverage-disqualification regimes.

\sectionestablished{
\item Stress tests explicitly target known weak points of the workflow.
\item Failures are treated as boundary information, not as noise to suppress.
\item The suite supports conservative refinement of assumptions and thresholds.
}
