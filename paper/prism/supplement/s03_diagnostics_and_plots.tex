\section{Diagnostics and Plot Conventions}
\label{supp:diagnostics}

\sectionpurpose{This section standardizes how calibration and posterior diagnostics are interpreted so scenario comparisons remain consistent.}

\sectionlearn{
\item Reliability and coverage diagnostics used in artifact generation.
\item Low-support and fallback flags that qualify interpretation.
\item How to read mixed scenario outcomes without overclaiming.
}

Diagnostics reported in scenario artifacts include reliability curves by model variant, coverage deltas at nominal 90\% and 95\% levels, and model-level degeneration flags where bin support is weak.

Interpretation rules used in this supplement:
\begin{itemize}[leftmargin=*]
\item Coverage-off-nominal flags are disqualifying for strong calibration claims.
\item Low-support flags trigger conservative interpretation of any pass result.
\item Fallback-based calibration variance implies wider posterior uncertainty by design.
\item Mixed outcomes in a family are not aggregated into unconditional success claims.
\end{itemize}

This convention follows a restrained Bayesian workflow perspective: diagnostics are part of the model output contract, not post-hoc annotations.

\sectionestablished{
\item Diagnostic interpretation rules are explicit and reproducible.
\item Support quality is a first-class gating signal for empirical claims.
\item Mixed evidence is carried forward without optimistic aggregation.
}
