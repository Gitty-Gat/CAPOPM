\section{Additional Proofs}
\label{supp:proofs}

\sectionpurpose{This section gives expanded arguments for the stability statements used in the main text.}

\sectionlearn{
\item A fuller perturbation decomposition for posterior means.
\item How concentration bounds enter constants in stability inequalities.
\item The role of Lipschitz correction assumptions in finite-sample controls.
}

\begin{proof}[Expanded proof sketch for the main stability theorem]
Write posterior mean as
\[
\mu(\alpha,\beta,y,n)=\frac{\alpha+y}{\alpha+\beta+n}.
\]
For two realizations, add and subtract an intermediate value using matched counts:
\[
|\mu-\mu'|\le |\mu(\alpha,\beta,y,n)-\mu(\alpha',\beta',y,n)| + |\mu(\alpha',\beta',y,n)-\mu(\alpha',\beta',y',n')|.
\]

First term: if corrected means differ by at most $L_t\Delta_p$ and concentration is bounded above and below, parameter perturbations are linear in $L_t\Delta_p$.

Second term: partial derivatives with respect to $(y,n)$ are bounded by inverse effective concentration factors, yielding constants multiplying $(\Delta_y+\Delta_n)$.

Combining terms gives
\[
|\mu-\mu'|\le C_t\left(L_t\Delta_p+\Delta_y+\Delta_n\right),
\]
with finite $C_t$ under bounded concentration.
\end{proof}

\begin{remark}
When concentration collapses toward zero because of aggressive uncertainty caps, constants increase. This is expected: PRISM chooses conservative uncertainty over artificial stability.
\end{remark}

\sectionestablished{
\item The stability bound follows by a direct decomposition argument.
\item Bounded concentration assumptions are explicit in the constant term.
\item Conservative concentration choices can widen sensitivity bounds.
}
