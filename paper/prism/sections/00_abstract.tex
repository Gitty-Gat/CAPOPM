\begin{abstract}
PRISM (Probabilistic Regime-Integrated Signal Model) is a Bayesian workflow for posterior inference on an event of interest using granular L3 order-flow information and calibrated machine-learning signals. The canonical pipeline is: L3 order flow $\rightarrow$ anchor beliefs $\rightarrow$ ML signal $\rightarrow$ calibration $\rightarrow$ fusion $\rightarrow$ corrections $\rightarrow$ posterior. The structural anchor prior is fixed at $\BetaDist(1,1)$, and machine-learning output is treated as a signal rather than a prior. Calibrated signal uncertainty is mapped conservatively into Beta parameters through moment matching with feasibility constraints and concentration caps. The workflow is designed for interpretable posterior summaries, stress-test diagnostics, and explicit limitation reporting under low-support or regime-instability conditions. The main paper provides the workflow, assumptions, and conservative theory claims; the supplement provides full synthetic design details, stress tests, diagnostics, and extended proofs.
\end{abstract}
