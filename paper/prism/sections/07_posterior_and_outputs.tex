\section{Posterior and Outputs}
\label{sec:posterior}

\sectionpurpose{This section defines the final posterior update and the reported output objects. The output layer is centered on probabilistic inference and diagnostics.}

\sectionlearn{
\item How corrected parameters combine with adjusted evidence.
\item Closed-form posterior summaries used in PRISM reports.
\item Which diagnostics must accompany posterior means.
}

Let $y_t^*$ denote corrected yes-count evidence and let $n_t^*$ denote corrected total count for window $t$ (with $0\le y_t^*\le n_t^*$). Given corrected prior parameters $(\alpha_{\mathrm{corr},t},\beta_{\mathrm{corr},t})$, PRISM updates through
\[
\pi_{\mathrm{post},t}(p)
=
\BetaDist\!\left(
\alpha_{\mathrm{corr},t}+y_t^*,
\beta_{\mathrm{corr},t}+n_t^*-y_t^*
\right).
\]

The posterior mean is
\[
p_{\mathrm{post},t} =
\frac{\alpha_{\mathrm{corr},t}+y_t^*}{\alpha_{\mathrm{corr},t}+\beta_{\mathrm{corr},t}+n_t^*}.
\]
The $(1-\eta)$ credible interval is obtained from Beta quantiles.

For every experiment scenario, posterior reporting includes:
\begin{itemize}[leftmargin=*]
\item posterior mean and interval width,
\item reliability-curve diagnostics,
\item Brier score and log score,
\item support flags (effective sample size, empty-bin checks, and calibration fallbacks).
\end{itemize}

\begin{remark}
PRISM outputs posterior beliefs about an event probability. Any downstream contract settlement or policy layer is external and must not be conflated with the posterior construction itself.
\end{remark}

\sectionestablished{
\item Posterior updating remains closed-form under corrected evidence.
\item Reported outputs are probabilistic summaries plus diagnostics.
\item The output contract is inference-first and explicitly separated from downstream decisions.
}
